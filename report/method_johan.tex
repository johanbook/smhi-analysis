% Method Johan
The code reads all data files from SMHI and for each file it creates an average for each year (including calculation standard deviation). Each file was plotted separately using a continuous error bars. By creating a semi-transparent overlay\footnote{This was done in Inkscape due to an incompatibility between OpenGL and ROOT.} a single graph vaguely describing general properties of the system, could be constructed.

The code utilized a class \texttt{treader} which parses a data file into a set of instances of the \texttt{tpoint}, each containing an entry consisting of date and temperature.

% Result Johan
The produced figure (see fig. ?) represents the distributions of average temperature for different years (y-axis) and different places (distinct colours). The darker areas correspond to regions shared by several places.

% Discussion
The top border does not show any direct patterns. However, the lower border does. Firstly, one can observe some form of cycle where the temperature is temporarily increased during a period roughly every five years

Secondly, one can see a slight hint of an overall temperature increase.
